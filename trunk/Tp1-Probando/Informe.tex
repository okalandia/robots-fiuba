% Tamaño de letra.
\documentclass[12pt,titlepage]{article}

%------------------------------ Paquetes ----------------------------------

% Paquetes:

%Para comentarios multilínea.
\usepackage{verbatim}

% Para tener cabecera y pie de página con un estilo personalizado.
\usepackage{fancyhdr}

% Codificación UTF-8
\usepackage[utf8]{inputenc}

% Castellano.
\usepackage[spanish]{babel}

% Tamaño de página y márgenes.
\usepackage[a4paper,headheight=16pt,scale={0.75,0.8},hoffset=0.5cm]{geometry}

% Para poder agregar notas al pie en tablas:
%\usepackage{threeparttable}

% Tipo de letra Helvetica (Arial).
%\usepackage{helvet}
%\renewcommand\familydefault{\sfdefault}

% Gráficos:

% Para incluir imágenes, el siguiente código carga el paquete graphicx
% según se esté generando un archivo dvi o un pdf (con pdflatex).

% Para generar dv.
%\usepackage[dvips]{graphicx}

% Para generar pdf.
\usepackage[pdftex]{graphicx}
\pdfcompresslevel=9

\usepackage{pdfpages}

%
% Directorio donde están las imagenes.
%
%\newcommand{\imgdir}{includes}
%\graphicspath{{\imgdir/}}

%------------------------------ ~paquetes ---------------------------------

%------------------------- Inicio del documento ---------------------------

\begin{document}

% ---------------------- Encabezado y pie de página -----------------------

% Encabezado: sección a la derecha.
% Pie de página: número de página a la derecha.

\pagestyle{fancy}
\renewcommand{\sectionmark}[1]{\markboth{}{\thesection\ \ #1}}
\lhead{}
\chead{}
\rhead{\rightmark}
\lfoot{}
\cfoot{}
\rfoot{\thepage}

% ---------------------- ~Encabezado y pie de página ----------------------

% -------------------------- Título y autor(es) ---------------------------

\title{ConcuShare}
\author{}

% -------------------------- ~Título y autor(es) --------------------------

% ------------------------------- Carátula --------------------------------

\begin{titlepage}

\thispagestyle{empty}

% Logo facultad más pie de la figura.
\begin{center}
\includegraphics[scale=0.55]{./Images/fiuba}\\
\large{\textsc{Universidad de Buenos Aires}}\\
\large{\textsc{Facultad De Ingeniería}}\\
\small{Año 2012 - 1\textsuperscript{er} Cuatrimestre}
\end{center}

\vfill

% Título central.
\begin{center}

\Large{\underline{\textsc{Sistema de Programaci\'on No Convencional de Robots}}}
\Large{\underline{\textsc{(75.70)}}}

\vfill

% Tabla de integrantes.

\Large{\underline{\textsc{Trabajo Pr\'actico}}}

\vfill

\Large\underline{Integrantes} \linebreak\linebreak

% Separación entre columnas.
\large\addtolength{\tabcolsep}{-3pt}
% Tres columnas con alineación centrada.
\begin{tabular}{|| c | c | c ||}
\hline
\textbf{Apellido, Nombre} & \textbf{Nro. Padrón} & \textbf{E-mail} \\
\hline
Bukaczewski, Verónica & 86954 & vero13@gmail.com \\
\hline
Rivero, Hern\'an & XXXXXX & riverohernanj@gmail.com \\
\hline
\end{tabular}
\end{center}

\vfill

\hrule
\vspace{0.2cm}

% Pie de página de la carátula.
\noindent\small{75.70 - Sistema de Programaci\'on No Convencional de Robots \hfill}

\end{titlepage}

% ------------------------------- ~Carátula -------------------------------

% -------------------------------- Índice ---------------------------------

% Hago que las páginas se comiencen a contar a partir de aquí.
\setcounter{page}{1}

% Índice.
\tableofcontents
\newpage

% -------------------------------- ~Índice --------------------------------

% ----------------------------- Inicio del tp -----------------------------


% Introducción.
\section{Objetivo}
El objetivo del presente trabajo pr\'actico es familiarizarnos con la herramienta Joone, utilizada para el estudio de Redes Neuronales. Y finalmente, poder realizar una an\'alisis de los resultados obtenidos.

\section{Preparando las corridas}
Se seleccion\'o la base de datos del Ta-Te-Ti, extra\'ida de la p\'agina UCI (Machine Learning Repository)



% ------------------------------ Fin del tp -------------------------------

\end{document}

%---------------------------- Fin del documento ---------------------------

